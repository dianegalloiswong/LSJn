\documentclass[11pt, openany]{article}
\usepackage[frenchb]{babel}
\usepackage[latin1]{inputenc}
\renewcommand{\labelitemi}{$\diamond$}



\bibliographystyle{ieeetr}  

\begin{document}



\noindent Semaine du 16 au 20 :

\

\noindent Lundi matin :

pr�sentation du LORIA, de l'�quipe Types, quelques consid�rations g�n�rales sur l'�tude de logiques et le calcul de s�quents par DG ; 

pr�sentation du sujet de stage par DL qui m'a donn� les articles \cite{LSJ} et \cite{Dyckhoff}, et m'a demand� dans un premier temps d'�tudier de pr�s la preuve de compl�tude et l'algorithme de \cite{LSJ}.

\

\noindent Lundi a-m - Mardi matin :

lecture des articles

impl�mentation en Caml de l'algorithme de \cite{LSJ} de fa\c con tr�s na�ve (utilisation de modules Set, ne profite pas de la propri�t� de la sous-formule (on recopie et retient toutes les formules !)... On parcourt donc � chaque fois les ensembles de formules enti�rement. Par exemple, lorsqu'� chaque appel r�cursif on commence par v�rifier si $\Gamma$ et $\Delta$ ont une intersection non vide, c'est tr�s long !), et de la construction d'un contre-mod�le � partir d'un arbre RJ (mais le premier algorithme ne permet pas de trouver un contre-mod�le minimal)

premiers tests qui semblent concluants, mais seulement sur de petites formules �crites � la main (tiers exclu, sa double n�gation, ...)

\

\noindent Mardi a-m, rencontre avec DL :

pr�sentation du syst�me $LSJ_{n}$ (?). me demande de prouver qu'il est �quivalent � $LSJ$, en me conseillant d'introduire une relation entre s�quents de $LSJ$ et de $LSJ_{n}$ pour exprimer l'hypoth�se de r�currence.

sugg�re le principe des tiroirs pour faire de plus gros tests, surtout que les arbres de d�rivation associ�s deviennent vite tr�s grands

\

\noindent Mardi a-m - Mercredi matin :

nouveaux tests : ``�quivalences cycliques'' et principe des tiroirs $\rightarrow$ correction de bugs

On voit rapidement les limites de cette impl�mentation :   tiroirs : 4 3 et 4 4 ok, 5 4 et 5 5 trop longs ; eq\_boucle : 19 et 20 : 1s pour LI, 10s pour LC ; 21 : LC trop long ; 26 : 10s pour LI

\

\noindent Mercredi a-m :

familiarisation avec le package bussproofs de LaTeX pour �crire des s�quents, premier aper\c cu de BibTeX, d�but du journal =)




\bibliographystyle{plain}
\bibliography{stage}



\end{document}
