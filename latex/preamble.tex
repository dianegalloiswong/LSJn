\documentclass[11pt, openany]{article}
%\usepackage{pstricks,pstricks-add,pst-math,pst-xkey}
\usepackage[frenchb]{babel}
%\usepackage{slashbox}
\usepackage{graphicx}
\usepackage{amsmath,amssymb,amstext}
\usepackage[latin1]{inputenc}
\usepackage[OT1]{fontenc}
%\usepackage{fancybox}
\usepackage{a4wide}
%\usepackage{fancyvrb}
\usepackage{pgf,tikz} %arbres
	\usetikzlibrary{cd}
%\usetikzlibrary{arrows}
\usepackage{xcolor}
%\usepackage{multicol}
%\usepackage[cm]{fullpage}
\usepackage{amsthm} %pour les th�or�mes sans num�ro
\usepackage{array,multirow,makecell}
\usepackage{mathrsfs}
\usepackage{floatflt}
\usepackage{verbatim}


\usepackage{bussproofs}\EnableBpAbbreviations


\bibliographystyle{ieeetr}  


\newcounter{moncompteur}
\newtheorem{q}[moncompteur]{\textbf{Question}}{}
\newtheorem{prop}[moncompteur]{\textbf{Proposition}}{}
\newtheorem{df}[moncompteur]{\textbf{D�finition}}{}
\newtheorem*{df*}{\textbf{D�finition}}{}
\newtheorem{rem}[moncompteur]{\textbf{Remarque}}{}
\newtheorem{theo}[moncompteur]{\textbf{Th�or�me}}{}
\newtheorem*{theo*}{\textbf{Th�or�me}}{}
\newtheorem{conj}[moncompteur]{\textbf{Conjecture}}{}
\newtheorem{cor}[moncompteur]{\textbf{Corollaire}}{}
\newtheorem{lm}[moncompteur]{\textbf{Lemme}}{}
\newtheorem*{lm*}{\textbf{Lemme}}{}
%\newtheorem{nota}[moncompteur]{ \textbf{Notation}}{}
%\newtheorem{conv}[moncompteur]{ \textbf{Convention}}{}
\newtheorem{exa}[moncompteur]{\textbf{Exemple}}{}
\newtheorem{ex}[moncompteur]{\textbf{Exercice}}{}
%\newtheorem{app}[moncompteur]{ \textbf{Application}}{}
%\newtheorem{prog}[moncompteur]{ \textbf{Algorithme}}{}
%\newtheorem{hyp}[moncompteur]{ \textbf{Hypoth�se}}{}
%\newenvironment{dem}{\noindent\textbf{Preuve}\\}{\flushright$\blacksquare$\\}
%\newenvironment{dem}{\noindent\textbf{Preuve.} }{\hfill$\blacksquare$\\}

%\newenvironment{myindentpar}[1]%
%{\begin{list}{}%
%         {\setlength{\leftmargin}{#1}}%
%         \item[]%
%}
%{\end{list}}


\renewcommand{\labelitemi}{$\bullet$}


%\newcommand{\R}{\mathbb{R}}
%\newcommand{\K}{\mathbb{K}}
%\newcommand{\N}{\mathbb{N}}
%\newcommand{\Z}{\mathbb{Z}}
%\newcommand{\C}{\mathbb{C}}
%\newcommand{\U}{\mathbb{U}}
%\newcommand{\Q}{\mathbb{Q}}
%\newcommand{\card}{\mathrm{card}}


\newcommand{\s}{\sigma}
%\newcommand{\l}{\lambda}
%\newcommand{\a}{\alpha}
%\newcommand{\r}{\rho}

%m�moire
\newcommand{\revAnd}{\rotatebox[origin=c]{180}{\&}}
\newcommand{\Ccal}{\mathcal{C}}
\newcommand{\Dcal}{\mathcal{D}}
\newcommand{\1}{\boldsymbol{1}}
\newcommand{\0}{\boldsymbol{0}}
\newcommand{\CProofs}{
%?\mathcal{C}?
\mathcal{CP}
}
%

\newcommand{\subsectionDescription}[1]{
\addtocontents{toc}{%\textit{
%\qquad\qquad
\hangindent=3.5\parindent \hangafter=0
\noindent
#1
}%}
}


\newcommand{\idees}[1]{
\textsf{(#1)}
}



%stage
\newcommand{\LK}{$\mathbf{LK}$}
\newcommand{\LJ}{$\mathbf{LJ}$}
\newcommand{\LJT}{$\mathbf{LJT}$}
\newcommand{\LSJ}{$\mathbf{LSJ}$}
\newcommand{\LSJn}{$\mathbf{LSJ\boldsymbol\ell}$}

%\newcommand{\T}{$\mathbf{T}$}
\newcommand{\T}{$\boldsymbol{T}$}

\newcommand{\Sig}{\mathfrak S}
\newcommand{\G}{\Gamma}
\newcommand{\D}{\Delta}
\newcommand{\Th}{\Theta}
\newcommand{\Gp}{\Gamma}
\newcommand{\Dp}{\Delta}
\newcommand{\Gt}{\widetilde\Gamma}
\newcommand{\Dt}{\widetilde\Delta}
\newcommand{\ClG}{\widetilde\Gamma}
\newcommand{\ClD}{\widetilde\Delta}


\newcommand{\imp}{\to\negthickspace}

\newcommand{\surj}{
%\mathrm{surj}
\boldsymbol
\Phi
}
\newcommand{\To}{\Rightarrow}
\newcommand{\forget}{\mathsf{forget}}



\newcommand{\autour}[5]{
%^{#2}_{#3}#1^{#4}_{#5}
%\scriptsize{#2}#1_{_{#4}}\scriptsize{#3}
%^{\mathbf{#2}}#1^{#3}_{_{#4}}
^{\mathbf{#2}\;}#1^{\;\mathit{#3}}_{_{#4}}
}





