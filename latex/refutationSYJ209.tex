\documentclass[11pt, openany]{article}
%\usepackage{pstricks,pstricks-add,pst-math,pst-xkey}
\usepackage[frenchb]{babel}
%\usepackage{slashbox}
\usepackage{graphicx}
\usepackage{amsmath,amssymb,amstext}
\usepackage[latin1]{inputenc}
\usepackage[OT1]{fontenc}
%\usepackage{fancybox}
\usepackage{a4wide}
%\usepackage{fancyvrb}
\usepackage{pgf,tikz}
\usetikzlibrary{arrows}


\newcounter{moncompteur}
\newtheorem{q}[moncompteur]{ \textbf{Question}}{}
\newtheorem{prop}[moncompteur]{ \textbf{Proposition}}{}
\newtheorem{df}[moncompteur]{ \textbf{D�finition}}{}
\newtheorem{rem}[moncompteur]{ \textbf{Remarque}}{}
\newtheorem{theo}[moncompteur]{ \textbf{Th�or�me}}{}
\newtheorem{conj}[moncompteur]{ \textbf{Conjecture}}{}
\newtheorem{cor}[moncompteur]{ \textbf{Corollaire}}{}
\newtheorem{lm}[moncompteur]{ \textbf{Lemme}}{}
%\newtheorem{nota}[moncompteur]{ \textbf{Notation}}{}
%\newtheorem{conv}[moncompteur]{ \textbf{Convention}}{}
\newtheorem{exa}[moncompteur]{ \textbf{Exemple}}{}
\newtheorem{ex}[moncompteur]{ \textbf{Exercice}}{}
%\newtheorem{app}[moncompteur]{ \textbf{Application}}{}
%\newtheorem{prog}[moncompteur]{ \textbf{Algorithme}}{}
%\newtheorem{hyp}[moncompteur]{ \textbf{Hypoth�se}}{}
\newenvironment{dem}{\noindent\textbf{Preuve}\\}{\flushright$\blacksquare$\\}
\newcommand{\cg }{[\kern-0.15em [}
\newcommand{\cd}{]\kern-0.15em]}
\newcommand{\R}{\mathbb{R}}
\newcommand{\K}{\mathbb{K}}
\newcommand{\N}{\mathbb{N}}
\newcommand{\Z}{\mathbb{Z}}
\newcommand{\C}{\mathbb{C}}
\newcommand{\U}{\mathbb{U}}
\newcommand{\Q}{\mathbb{Q}}
\newcommand{\card}{\mathrm{card}}


%stage
\newcommand{\G}{\Gamma}
\newcommand{\D}{\Delta}
\newcommand{\Th}{\Theta}

\newcommand{\To}{\Rightarrow}

\usepackage{bussproofs}
\newcommand{\ze}{\AxiomC}
\newcommand{\rien}{\AxiomC{}}
\newcommand{\un}{\UnaryInfC}
\newcommand{\bi}{\BinaryInfC}
\newcommand{\tr}{\TrinaryInfC}
\newcommand{\lab}{\RightLabel}

\EnableBpAbbreviations

%\AxiomC{form } \UnaryInfC{form } \BinaryInfC{form } \TrinaryInfC{form }

\begin{document}
\renewcommand{\labelitemi}{$\diamond$}
\vspace{2cm}~~\\

%fof(axiom1,axiom,(
%( ( ( p1 & ( p2 & p3 ) ) | ( ( ~(~(p1)) => f)  | ( ( p2 => f)  | ( p3 => f)  ) ) ) => f) )).
%
%fof(con,conjecture,(
%f
%)).


\begin{prooftree}
	\AXC	{}
	
	\RightLabel{Irr}
	\UIC{$1:f \;,\; 0: p2 \;,\; 0: p3 \;,\; 2: \bot \;,\; 1: p1 \quad\To_{1}\quad 0:  p1  \;,\; 0: f  \;,\;  0: f  \;,\;  0: f  \;,\; 0: f \;,\; 1: \bot $}

	\RightLabel{$\to R$}
	\UIC{$1:f \;,\; 0: p2 \;,\; 0: p3 \;,\; 2: \bot \quad\To_{1}\quad 0:  p1  \;,\; 0: f  \;,\;  0: f  \;,\;  0: f  \;,\; 0: f \;,\; 1: \lnot p1$}

	\RightLabel{Succ}
	\UIC{$1:f \;,\; 0: p2 \;,\; 0: p3 \;,\; 0: \lnot\lnot p1 \quad\To_{0}\quad 0:  p1  \;,\; 0: f  \;,\;  0: f  \;,\;  0: f  \;,\; 0: f$}

	\RightLabel{$\to R$}
	\UIC{$1:f \;,\; 0: p2 \;,\; 0: p3 \quad\To_{0}\quad 0:  p1  \;,\; 0: \lnot\lnot p1 \to f  \;,\;  0: f  \;,\;  0: f  \;,\; 0: f$}

	\RightLabel{$\to R$}
	\UIC{$1:f \;,\; 0: p2 \quad\To_{0}\quad 0:  p1  \;,\; 0: \lnot\lnot p1 \to f  \;,\;  0: f  \;,\;  0: p3 \to f  \;,\; 0: f$}

	\RightLabel{$\to R$}
	\UIC{$1:f \quad\To_{0}\quad 0:  p1  \;,\; 0: \lnot\lnot p1 \to f  \;,\;  0: p2 \to f  \;,\;  0: p3 \to f  \;,\; 0: f$}

	\RightLabel{$\land R1$}	
	\UIC{$1:f \quad\To_{0}\quad 0:  p1 \land ( p2 \land p3 )  \;,\; 0: \lnot\lnot p1 \to f  \;,\;  0: p2 \to f  \;,\;  0: p3 \to f  \;,\; 0: f$}

	\RightLabel{$\lor R \,(\times3)$}	
	\UIC{$1:f \quad\To_{0}\quad 0: (\;( p1 \land ( p2 \land p3 ) ) \lor (\;( \lnot\lnot p1 \to f)  \lor (\; ( p2 \to f)  \lor ( p3 \to f)  \;) ) )\;,\; 0: f$}

	\RightLabel{$\to L2$}	
	\UIC{$0:[ \; (\;( p1 \land ( p2 \land p3 ) ) \lor (\;( \lnot\lnot p1 \to f)  \lor (\; ( p2 \to f)  \lor ( p3 \to f)  \;) ) ) \; ] \to f  \quad\To_{0}\quad 0: f$}

	\RightLabel{$\to R$}	
	\UIC{$\To_{0}\quad 0:( \;[ \; (\;( p1 \land ( p2 \land p3 ) ) \lor (\;( \lnot\lnot p1 \to f)  \lor (\; ( p2 \to f)  \lor ( p3 \to f)  \;) ) ) \; ] \to f \; ) \to f$}
	
\end{prooftree}

















\end{document}


 \AxiomC{}
     \UnaryInfC{D}
     \DisplayProof


\begin{prooftree}
     \ax{A}
     \AxiomC{B}
%\RightLabel{ici}
     \BinaryInfC{D}
     \end{prooftree}

 \begin{prooftree}
          \AxiomC{A}
 \RightLabel{1}
          \AxiomC{B}
          \AxiomC{C}
 \RightLabel{2}
          \BinaryInfC{D}
 \RightLabel{3}
          \BinaryInfC{E}
 \RightLabel{4}
     \end{prooftree}






